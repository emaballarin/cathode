\documentclass{beamer}

\usepackage[utf8]{inputenc}
\usepackage[english]{babel}

\usepackage{graphicx}
\usepackage{amsmath}
\usepackage{mathtools}
\usepackage{wrapfig}
\usepackage{amsfonts}
\usepackage{csquotes}
\usepackage{epigraph}
\usepackage{float}
\usepackage{lipsum}
\usepackage{blindtext}
\usepackage{multirow}
\usepackage[ruled,vlined]{algorithm2e}
\usepackage{bm}
\usepackage{xcolor}
\usepackage{upgreek}
\usepackage{calc}
\usepackage{lscape}

% TYPESETTING SUGAR
\DeclareMathOperator*{\argmax}{argmax}
\DeclareMathOperator*{\argmin}{argmin}
\DeclareMathOperator*{\Find}{Find}
\DeclareMathOperator*{\find}{find}
\DeclareMathOperator*{\NNet}{NNet}
\DeclareMathOperator*{\EGR}{EGR}
\def\leq{\leqslant}
\def\geq{\geqslant}


\usetheme{Madrid}

%-----------------------------------------------%
%-----------------------------------------------%



%
% TITLE SLIDE
%

\makeatletter
\setbeamertemplate{footline}{%
  \leavevmode%
  \hbox{\begin{beamercolorbox}[wd=.4\paperwidth,ht=2.5ex,dp=1.125ex,leftskip=.20cm plus1fill,rightskip=.20cm]{author in head/foot}%
    \usebeamerfont{author in head/foot}\insertshortauthor
  \end{beamercolorbox}%
  \begin{beamercolorbox}[wd=.6\paperwidth,ht=2.5ex,dp=1.125ex,leftskip=.20cm,rightskip=.20cm plus1fil]{title in head/foot}%
    \usebeamerfont{title in head/foot}\insertshorttitle \hspace*{1.2ex}
    \insertframenumber{}/\inserttotalframenumber
  \end{beamercolorbox}}%
  \vskip0pt%
}
\makeatother

\title[\textit{Neural ODEs} for \textit{real-world}, irregularly-sampled, time series prediction]{\textit{Neural ODEs} \\ for \textit{real-world}, irregularly-sampled, time series prediction}

\author[E. Ballarin, M. N. Plasencia Palacios, A. Tasciotti]{Emanuele Ballarin \\ Milton Nicolás Plasencia Palacios \\ Arianna Tasciotti}
\institute[]{Department of Mathematics and Geosciences, University of Trieste}
\date[]{\textit{Statistical Machine Learning} course \\ Final Project $\sim$ June 2020}

\titlegraphic{\includegraphics[width=1.25cm,height=1.3cm,keepaspectratio]{dssclogo.png}
	\hspace*{93.04mm}~% Horizontal space is tailored for 1.3cm square logos
	\includegraphics[width=1.3cm,height=1.25cm,keepaspectratio]{logouniv.png}
}

%-----------------------------------------------%
%-----------------------------------------------%



%
% ADAPTIVE TABLE OF CONTENTS
%

\AtBeginSection[]
{
	\begin{frame}
		\frametitle{\textit{We are here...}}
		\tableofcontents[currentsection]
	\end{frame}
}

%-----------------------------------------------%
%-----------------------------------------------%



%
% DOCUMENT
%

\begin{document}

% Insert title slide
\frame{\titlepage}

%-----------------------------------------------%

%% TIME SERIES
%\section{Introduzione al paradigma del \textit{Deep Learning}}{
\section{\textit{Theory first!}}{

%% SLIDE 1.1:
\begin{frame}
\frametitle{time series}

\end{frame}
%---------------------------------------------%
%% SLIDE 1.2:
\begin{frame}
	\frametitle{rnn}
	%-----------------------------------------%
		%-----------------------------------------%
\end{frame}
%---------------------------------------------%
%% SLIDE 1.3:
\begin{frame}
	\frametitle{gru}
\end{frame}
%---------------------------------------------%

%% SLIDE 1.4:
\begin{frame}
	\frametitle{resnet}

\end{frame}
%---------------------------------------------%
%% SLIDE 1.5:
\begin{frame}
	\frametitle{ode}
\end{frame}
%---------------------------------------------%
%% SLIDE 1.6:
\begin{frame}
	\frametitle{rnnode}
\end{frame}
%-----------------------------------------------%
%% SLIDE 1.7:
\begin{frame}
	\frametitle{latent ode}
\end{frame}
%---------------------------------------------%
%% SLIDE 1.8:
\begin{frame}
	\frametitle{vae}
\end{frame}
%---------------------------------------------%
%% SLIDE 1.9:
\begin{frame}
	\frametitle{rnnode+ latent ode+vae}
\end{frame}
}

\section{\textit{Deep learning works great, except when it doesn't.}}{
%---------------------------------------------%
%% SLIDE 1.10:
\begin{frame}
	\frametitle{layer/batchnorm}
\end{frame}
%---------------------------------------------%
%% SLIDE 1.11:
\begin{frame}
	\frametitle{gradient clipping}
\end{frame}
%---------------------------------------------%
%% SLIDE 1.12:
\begin{frame}
	\frametitle{optim adamw}
\end{frame}
%-----------------------------------------------%
%% SLIDE 1.13:
\begin{frame}
	\frametitle{loss l1}
\end{frame}
}

\section{\textit{No data, no learning!}}{
%---------------------------------------------%
%% SLIDE 1.14:
\begin{frame}
	\frametitle{dati}
\end{frame}
%---------------------------------------------%
%% SLIDE 1.15:
\begin{frame}
	\frametitle{dati}
\end{frame}
%-----------------------------------------------%
%% SLIDE 1.16:
\begin{frame}
	\frametitle{finestre}
\end{frame}
%---------------------------------------------%
%% SLIDE 1.17:
\begin{frame}
	\frametitle{batch}
\end{frame}
}

\section{\textit{So, what?}}{
%---------------------------------------------%
%% SLIDE 1.18:
\begin{frame}
	\frametitle{risultati}
\end{frame}
%---------------------------------------------%
%% SLIDE 1.19:
\begin{frame}
	\frametitle{accuracy}
	\end{frame}
%---------------------------------------------%
%% SLIDE 1.20:
\begin{frame}
	\frametitle{conclusioni}
\end{frame}
}
%% THANKS!
\begin{frame}
	\frametitle{Greetings}
	Thank you for your attention.
\end{frame}

%-----------------------------------------------%
%-----------------------------------------------%
\end{document}